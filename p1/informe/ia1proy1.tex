\documentclass[letterpaper,12pt, titlepage]{article}

\usepackage[spanish]{babel}
\usepackage[utf8]{inputenc}

\usepackage{graphicx}
\usepackage{tabularx}
\usepackage{slashbox}
\usepackage[intlimits]{amsmath}
\usepackage{amssymb}
%\usepackage[left=2cm, right=2cm, top=4cm, bottom=3.5cm]{geometry}

\DeclareGraphicsExtensions{.jpg,.pdf,.mps,.png,.eps}

\newcommand{\y}{\ \land \ }
\renewcommand{\o}{\ \lor \ }
\newcommand{\no}{\neg}
\newcommand{\eq}{\equiv}
\newcommand{\nq}{\not\equiv}
\newcommand{\impl}{\Rightarrow}
\newcommand{\conseq}{\Leftarrow}
\newcommand{\true}{\textit{true}}
\newcommand{\false}{\textit{false}}
\newcommand{\existe}[3]{(\exists #1 \mid #2 : #3)}
\newcommand{\existee}[4]{(\exists !_{#1} #2 \mid #3: #4)}

\newcommand{\patodo}[3]{(\forall #1 \mid #2 : #3)}
\newcommand{\cuant}[3]{(* #1 \mid \ #2 : #3)}

\usepackage{setspace}
\onehalfspacing

%Bases de datos
\newcommand{\ms}{\texttt}

\begin{document}

\title{	\includegraphics[height=80pt]{usb.jpg} \\
CI-5437 \\ Inteligencia Artificial I \\
Proyecto I\\
Búsqueda}
\author{Kelwin Fernández y Alejandro Machado\\
	Universidad Simón Bolívar} 
\date{Junio de 2010} 
\maketitle

\section{Decisiones de implementación}
La programación concurrente es un elemento importante como mecanismo para optimizar
el desempeño de aplicaciones virtuales. En el presente trabajo se lleva a cabo una investigación
que busca mejorar el rendimiento de un programa sencillo, que calcula el valor de la integral definida 
de un polinomio de tercer grado, a través de técnicas de programación concurrentes estudiadas en
la asignatura de Sistemas Operativos. 

\begin{enumerate}
	\item
\end{enumerate}

\section{Discusión de resultados}

\section{Recomendaciones}

\end{document}