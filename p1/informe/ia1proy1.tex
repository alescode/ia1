\documentclass[letterpaper,12pt, titlepage]{article}

\usepackage[spanish]{babel}
\usepackage[utf8]{inputenc}

\usepackage{graphicx}
\usepackage{tabularx}
\usepackage{slashbox}
\usepackage[intlimits]{amsmath}
\usepackage{amssymb}
%\usepackage[left=2cm, right=2cm, top=4cm, bottom=3.5cm]{geometry}

\DeclareGraphicsExtensions{.jpg,.pdf,.mps,.png,.eps}

\newcommand{\y}{\ \land \ }
\renewcommand{\o}{\ \lor \ }
\newcommand{\no}{\neg}
\newcommand{\eq}{\equiv}
\newcommand{\nq}{\not\equiv}
\newcommand{\impl}{\Rightarrow}
\newcommand{\conseq}{\Leftarrow}
\newcommand{\true}{\textit{true}}
\newcommand{\false}{\textit{false}}
\newcommand{\existe}[3]{(\exists #1 \mid #2 : #3)}
\newcommand{\existee}[4]{(\exists !_{#1} #2 \mid #3: #4)}

\newcommand{\patodo}[3]{(\forall #1 \mid #2 : #3)}
\newcommand{\cuant}[3]{(* #1 \mid \ #2 : #3)}

\usepackage{setspace}
\onehalfspacing

%Bases de datos
\newcommand{\ms}{\texttt}

\begin{document}

\title{	\includegraphics[height=80pt]{usb.jpg} \\
CI-5437 \\ Inteligencia Artificial I \\
Proyecto I\\
Búsqueda}
\author{Kelwin Fernández y Alejandro Machado\\
	Universidad Simón Bolívar} 
\date{Junio de 2010} 
\maketitle

\section{Decisiones de implementación}
\begin{itemize}
\item Representación de estado (caso bfs y caso ida*)

\item Explicación de obtener sucesor (ambos casos)

\item Detalles de agregar preferencia (en el sentido
de hacer una agregacion logaritmica en ciertos casos)
\item Nodos generados almacenados en un vector ordenados
(búsqueda logarítmica)

\item Explicar mejora de considerar que para los hijos
de un mismo padre se computa este una sola vez.

\item Se precomputa para cada nodo del bfs una funcion de
clasificacion que permite no tener que generar todos los
estados de la tabla de visitados para comparar con el
nuevo a estado a expandir. Presentar informalmente
ciertos resultados que se obtuvieron con respecto a las
proporciones.

\end{itemize}

\section{Discusión de resultados}

\section{Recomendaciones}

\end{document}