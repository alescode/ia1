\documentclass[letterpaper,12pt, titlepage]{article}

\usepackage[spanish]{babel}
\usepackage[utf8]{inputenc}

\usepackage{graphicx}
\usepackage{tabularx}
\usepackage{slashbox}
\usepackage[intlimits]{amsmath}
\usepackage{amssymb}
%\usepackage[left=2cm, right=2cm, top=4cm, bottom=3.5cm]{geometry}

\DeclareGraphicsExtensions{.jpg,.pdf,.mps,.png,.eps}

\begin{document}

\title{	\includegraphics[height=80pt]{usb.jpg} \\
CI-5437 \\ Inteligencia Artificial I \\
Proyecto I\\
Búsqueda}
\author{Kelwin Fernández y Alejandro Machado\\
	Universidad Simón Bolívar} 
\date{Junio de 2010} 
\maketitle

\section{Decisiones de implementación}

\subsection*{Representación de estados}

\subsubsection*{BFS}
   Como se cuenta con un máximo de 250 candidatos y un número muy elevado
   de posibles preferencias distintas, se ha escogido una representación compacta 
   para cada estado, lo que permitirá ahorrar recursos para que el algoritmo
   pueda concluir su ejecución, en ciertos casos, sin agotar la memoria del computador.

   En lugar de almacenar un perfil completo, para cada estado
   se guarda cuál fue el último cambio elemental realizado (cuál candidato, en cuál
   preferencia), y un apuntador al estado padre.

    Es necesario mantener una lista de estados ``cerrados" (ya visitados por el
    algoritmo) para una correcta implementación de búsqueda en amplitud.
    En este caso, los estados visitados se mantuvieron en un vector ordenado
    por un entero asociado a cada estado, que forma también parte de su representación.

    Este atributo de cada nodo es calculado con una función de clasificación, que
    se obtuvo a partir de la heurística sugerida para el algoritmo IDA*. La intuición
    detrás de esta decisión es la siguiente:
     dos estados que evalúan a un diferente valor de la función de clasificación
    deben ser distintos, y por lo tanto no hay que obtener los perfiles asociados
    y compararlos, lo cual consume tiempo. En algunos casos, una búsqueda binaria
    sobre el vector de nodos visitados puede arrojar el resultado que se necesita
    sin ningún cómputo adicional.

    Finalmente, también se almacena la profundidad del estado, a fin de no seguir
    expandiendo fronteras del último nivel si ya se ha hallado una solución
    en éste.

\subsubsection*{IDA*}
    Para este algoritmo se utilizó un solo perfil, y sobre éste se aplican
    y desaplican cambios elementales para explorar el espacio de estados.

    Explicar lista de cerrados.

\subsection*{Obtención de sucesores}
\subsubsection*{BFS}
    Para obtener los sucesores de un estado, debe construirse primero
    el perfil asociado y luego aplicar todos los posibles cambios elementales
    sobre éste.

\subsubsection*{IDA*}
    Se aplica un cambio elemental sobre
    el perfil actual, y se explora el perfil hijo. Posteriormente,
    se deshace este cambio elemental; este proceso se repite para
   cada transición posible.

\subsection*{Optimizaciones}

\subsubsection*{General}
    \begin{enumerate}
       \item Agregar un votante a una preferencia existente dentro de un perfil
       es $O(log(n))$, donde $n$ es el número de preferencias. Esto se logra
       manteniendo las preferencias ordenadas dentro de cada perfil.
    \end{enumerate}

\subsection*{BFS}
    \begin{enumerate}
       \item Los perfiles visitados se mantienen en un vector ordenado,
       lo que garantiza, al utilizar búsqueda binaria, que a lo sumo
       se realizará un número logarítmico (en el número de estados) de
       comparaciones para determinar si un estado ha sido visitado.
       \item En principio, los perfiles se deben generar
    
        - Optimizaciones:
            BFS:
                - Mantener dos perfiles: padre y un hijo, para no recomputar el padre en toda una frontera
                - Se guarda el nivel de profundidad para no expandir nodos más allá de una meta
                - Función de clasificación: dado que es una funcion, si dos nodos son iguales sus funciones serán iguales, ahorrar comparaciones pesadas.

/bin/bash: xclip: command not found

\begin{itemize}
\item Nodos generados almacenados en un vector ordenados
(búsqueda logarítmica)

\item Explicar mejora de considerar que para los hijos
de un mismo padre se computa este una sola vez.

\item Se precomputa para cada nodo del bfs una funcion de
clasificacion que permite no tener que generar todos los
estados de la tabla de visitados para comparar con el
nuevo a estado a expandir. Presentar informalmente
ciertos resultados que se obtuvieron con respecto a las
proporciones.

\end{itemize}

\section{Discusión de resultados}
Mejor no memorizar
\section{Recomendaciones}

\end{document}
