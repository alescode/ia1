\documentclass[letterpaper,12pt, titlepage]{article}

\usepackage[spanish]{babel}
\usepackage[utf8]{inputenc}

\usepackage{pdflscape}
\usepackage{graphicx}
\usepackage{tabularx}
\usepackage{slashbox}
\usepackage[intlimits]{amsmath}
\usepackage{amssymb}
%\usepackage[left=2cm, right=2cm, top=4cm, bottom=3.5cm]{geometry}

\DeclareGraphicsExtensions{.jpg,.pdf,.mps,.png,.eps}

\newcommand{\ms}{\texttt}

\newcommand{\gra}[2]{
	\includegraphics[height=#2cm]{#1}
}

\newcommand{\grac}[2]{
	\begin{center}
	\includegraphics[height=#2cm]{#1}
	\end{center}
}

\begin{document}
\setcounter{section}{5}
\setcounter{page}{7}

En base a estos resultados, puede afirmarse que aproximadamente la mitad
de las veces la función de clasificación evita que se realicen comparaciones
(costosas) entre perfiles para revisar si uno ha sido visitado. Sin embargo,
la tendencia parece ser que problemas cuya solución óptima se logra con
más cambios elementales tienen valores medios de proporción más bajos que
problemas más ``fáciles".

En general, los problemas son resueltos más rápidamente utilizando IDA*
que usando BFS. Este resultado sugiere que la heurística utilizada $T'(x)$
fue una buena elección; valiéndose de algún mecanismo como tablas
de transposición podrían lograrse mejores tiempos de solución.

\section{Recomendaciones}
Un enfoque diferente que podría haberse tomado al enfrentarse el problema
de modelar los estados es el de representar las preferencias con
enteros que codifiquen cada una de las posibles permutaciones de candidatos.
Para esto sería necesaria una librería o lenguaje de programación
que maneje números enteros arbitrariamente grandes.

En este sentido, sería interesante explorar qué
tanto pueden mejorarse los resultados utilizando un algoritmo eficiente capaz de codificar permutaciones
en números y viceversa.

Un algoritmo con estas características está descrito en \cite{1}.

\begin{thebibliography}{9}

\bibitem{1}
    Bonet, B.
    \emph{Efficient Algorithms to Rank and Unrank Permutations in Lexicographic Order}.
    AAAI-Workshop on Search in AI and Robotics.
    2008, Universidad Simón Bolívar, Venezuela.
\end{thebibliography}

\end{document}
